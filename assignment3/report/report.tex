\documentclass{article}[40pt]
\usepackage{ucs}
\usepackage{subcaption}
\usepackage{graphicx}
\usepackage{float}
\usepackage{grffile}
\usepackage[utf8x]{inputenc}
\usepackage[greek,english]{babel}
\usepackage{xspace}
\usepackage{alphabeta}
\usepackage{amssymb}
\usepackage[colorlinks]{hyperref}
\usepackage[fleqn]{mathtools}
\usepackage{adjustbox}
\usepackage{subcaption}
\usepackage{xcolor}
\usepackage{listings}
\hypersetup{
    colorlinks=false,% make the links colored
}

\definecolor{codegreen}{rgb}{0,0.6,0}
\definecolor{codegray}{rgb}{0.5,0.5,0.5}
\definecolor{codepurple}{rgb}{0.58,0,0.82}
\definecolor{backcolour}{rgb}{0.95,0.95,0.92}

\lstdefinestyle{mystyle}{
    backgroundcolor=\color{backcolour},   
    commentstyle=\color{codegreen},
    keywordstyle=\color{magenta},
    numberstyle=\tiny\color{codegray},
    stringstyle=\color{codepurple},
    basicstyle=\ttfamily\footnotesize,
    breakatwhitespace=false,         
    breaklines=true,                 
    captionpos=b,                    
    keepspaces=true,                 
    numbers=left,                    
    numbersep=5pt,                  
    showspaces=false,                
    showstringspaces=false,
    showtabs=false,                  
    tabsize=2
}

\title{ΠΑΝΕΠΙΣΤΗΜΙΟ ΙΩΑΝΝΙΝΩΝ \\ ΤΜΗΜΑ ΜΗΧΑΝΙΚΩΝ Η/Υ  \&  ΠΛΗΡΟΦΟΡΙΚΗΣ \\  ΜΥΕ041 - ΔΙΑΧΕΙΡΙΣΗ ΣΥΝΘΕΤΩΝ ΔΕΔΟΜΕΝΩΝ \\ ΕΡΓΑΣΙΑ 3}
\author{Κίτσιος Κωνσταντίνος 4388 }
\begin{document}
\maketitle
\newpage
\tableofcontents
\newpage
\section{Πηγαίος κώδικας}
Η εργασία έχει ένα αρχείο πηγαίου κώδικα, το top\_k.py. Θα πρέπει τα αρχεία του dataset να βρίσκονται στον ίδιο φάκελο με τον πηγαίο κώδικα. Η εκτέλεση του προγράμματος γίνεται με την εντολή:\\
python3 top\_k.py k, όπου k ο αριθμός των αντικειμένων προς εμφάνιση.   
\section{MinHeap}
Η κλάση MinHeap αναπαριστά ένα σωρό minheap, ο οποίος είναι μια δομή δεδομένων που χρησιμοποιείται για την αποθήκευση στοιχείων με συγκεκριμένη σειρά και ταχεία πρόσβαση στο ελάχιστο στοιχείο. Σε έναν min heap, το ελάχιστο στοιχείο βρίσκεται πάντα στην κορυφή του σωρού (το πρώτο στοιχείο της λίστας heap).\\
Ο κώδικας ξεκινά με τη δήλωση της κλάσης MinHeap και την αρχικοποίηση του σωρού heap ως μια κενή λίστα μέσω της μεθόδου \_\_init\_\_.\\
Η μέθοδος insert χρησιμοποιείται για την εισαγωγή ενός στοιχείου στον σωρό. Παίρνει δύο ορίσματα, το score που αντιπροσωπεύει την τιμή του στοιχείου και το ID που αναγνωρίζει μοναδικά το στοιχείο. Το στοιχείο προστίθεται στο τέλος της λίστας heap και μετά ακολουθεί η μέθοδος \_bubble\_up που εξισορροπεί το σωρό μετακινώντας το νέο στοιχείο προς τα πάνω μέχρι να βρει τη σωστή του θέση. Αυτό γίνεται συγκρίνοντας την τιμή του στοιχείου με την τιμή του γονέα του και αν είναι μικρότερη, γίνεται ανταλλαγή των θέσεων τους. Ο έλεγχος γίνεται επαναληπτικά μέχρι το στοιχείο να βρει τη σωστή του θέση στον σωρό.\\
Η μέθοδος extract\_min χρησιμοποιείται για την αφαίρεση και επιστροφή του ελάχιστου στοιχείου από τον σωρό. Αν ο σωρός είναι άδειος, επιστρέφεται None. Το ελάχιστο στοιχείο βρίσκεται πάντα στην κορυφή του σωρού, οπότε αποθηκεύεται στη μεταβλητή min\_item. Έπειτα, το τελευταίο στοιχείο της λίστας heap αφαιρείται και αποθηκεύεται στη μεταβλητή last\_item. Αν ο σωρός δεν είναι άδειος, το last\_item τοποθετείται στην κορυφή του σωρού και μετά ακολουθεί η μέθοδος \_sink\_down που εξισορροπεί τον σωρό μετακινώντας το στοιχείο προς τα κάτω μέχρι να βρει τη σωστή του θέση. Αυτό γίνεται συγκρίνοντας την τιμή του στοιχείου με τις τιμές των αριστερών και δεξιών παιδιών του και αν είναι μικρότερη από τις δύο, γίνεται ανταλλαγή των θέσεων τους. Ο έλεγχος γίνεται επαναληπτικά μέχρι το στοιχείο να βρει τη σωστή του θέση στον σωρό.
\newpage
\section{Κυρίως πρόγραμα}
Το κυρίως πρόγραμμα, αρχικά διαβάζει τα περιεχόμενα του αρχείου rnd.txt και τα αποθηκεύει σε ένα λεξικό, όπου το κλειδί είναι το ID του αντικειμένου που διαβάζεται και τιμή το score του. Έπειτα ξεκινάει το εναλλάξ διάβασμα των δύο αρχείων seq1.txt \& seq2.txt. Για κάθε γραμμή του αρχείου seq1.txt που διαβάζεται γίνεται η εξής διαδικασία:\\
Διαβάζεται το ID του αντικείμένου και το score του, γίνεται ενημέρωση του κάτω ορίου, του threshold και του τελευταίου score που έχει διαβαστεί μέσω της συνάρτησης update\_lower\_bound. Έπειτα καλείται η συνάρτηση update\_Wk η οποία ενημερώνει τον σωρό, αφού όμως έχουν βρεθεί τα πρώτα k αντικείμενα. Έπειτα γίνεται έλεγχος για τερματισμό. Η ίδια διαδικασία γίνεται και για κάθε γραμμή του αρχείου seq2.txt που διαβάζεται. Στο τέλος εκτυπώνονται τα ζητούμενα αποτελέσματα.
\section{Προβλήματα}
Ο συγκεκριμένος πηγαίος κώδικας, για k = 5 εμφανίζει σωστά τα πρώτα 4 αντικείμενα και όχι τα 5. Όταν έτρεξα το πρόγραμα για k = 10, τα πρώτα 5 αντικείμενα ήταν σωστά και τα υπόλοιπα ήταν λάθος. Έχω δει όλη τη λογική του προγράμματος και θεωρώ πως είναι η σωστή. Ίσως να ευθύνεται ο έλεγχος για τον τερματισμό του αλγόριθμου. Δεν κατάφερα να βρω τη λύση. 
\end{document}